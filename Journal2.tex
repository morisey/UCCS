\documentclass{article}
%\documentclass[journal]{IEEEtran}
%\usepackage[style=ieee]{biblatex}
\usepackage[numbers]{natbib}
\usepackage[english]{babel}
\usepackage[margin=1.0in]{geometry}
\usepackage{amsmath}
\usepackage{textcomp}
\usepackage{graphicx}
\usepackage{hyperref}
\usepackage{csquotes}
\usepackage{type1cm}
\usepackage{lettrine}
\newcommand{\myname}{Jennifer S. Smith}
\title{Journal 2: Descent into Nature}
\author{\myname\\
Department of Computer Science\\School of Engineering and Applied Science\\University of Colorado at Colorado Springs\\Colorado Springs, Colorado 80918\\Email: \href{mailto:jsmith@uccs.edu}{jsmith60@uccs.edu}\\
\href{https://github.com/morisey/UCCS}{GitHub Root}
}
\begin{document}
\maketitle
\section{Process}
\lettrine[findent=2pt]{\fbox{\textbf{I}}}{ }chose Nature \cite{nature} because it was always the Holy Grail of publications in my past life. It has articles on a variety of topics and currently is the highest ranked journal at Google Scholar with a h5 of 368. While the articles may not all specifically apply to research I would do, there are methods, modes of thought, and writing styles that could be adopted. The journal also has plain language versions of many articles so that those in other fields can get the story even if they are unfamiliar with the science. Reading this journal helps keep the larger context of science and politics around the world. Ethics play a large role in this issue.

To build the bibliography I used the Kraemer Library search to single out a specific issue. At this time I only know how to export them to Zotero one by one and this involves downloading single .ris files, creating a library, then double clicking the file to import. Then I export the edited list (in case other files sneaked into the list) to bibtex, and finally uploading to Overleaf. Overleaf did not like the lengthy URLs and when I tried to comment them out that created another error so I deleted them. Google Scholar has a nice interface for popping up the bibtex entry instead of downloading it and that is nice for building a quick bib. I can export it later to Z

I tried using Google Scholar and Nature's website to figure out a way to pull the bibliography as one unit but had no luck.

After this I followed the instructions from Canvas. I have been skimming and deep reading papers for 25 years and not much has changed. For my homeland security courses at UCCS I've been reading multiple papers and summarizing them in 3-600 words on a weekly basis for the last year. The papers in Nature are a welcome return to science from social science. It is interesting to contrast the styles.

Reading an entire journal felt more awkward than reading a collection of articles. Mostly because I was trying to follow along in the bibliography and make notes. It's nice that Zotero keeps my proxy happy and I can actually bypass the library links and stay at the journal site. Nature makes it easy to scan because it front loads the good stuff and the methods and supplementary material are tacked at the end. The diagrams are also easy to understand because they are full color and have long explanations. When someone submits to Nature they are going all out with their best work. The international representation is also heavily felt.

After finishing the mechanics of the assignment I felt like it was much more laborious than my old method of making notes on printouts of articles and dropping them into folders. My mental organization is based on the time and context in which I read something, I don't like all the new overhead but it will be worth it. The assignment felt cumbersome and not aligned well with what was in the lecture s I spent a lot of time just staring at the assignment trying to figure out what to do and what not to do. If I had just done it as described in the lecture I would have been much more effective. Most of the critical and creative elements were things I have always done I just never gave it any thought. Thinking in a structured manner is very uncomfortable. I think I would rather do critical and creative the way it comes naturally first then consulting the instructions to catch a few more things would be more effective for me in the future.
\section{Classroom Read}
A skim of the three articles gives the following (citations not included):\\
Vikhe \textit{et al.}: A quick glance shows a virtually unreadable paper \cite{vikhe_security_2015} that looks like it was generated by and AI. At best it was translated by a program but given how common usage of English is in India, surely it would still read better. The journal looks sketchy at best. Skip

The second is an article by a student and a published doctoral candidate (Pal and Anand \cite{pal_crypt_2018}). The review dates look a bit quick but this is better written so let's look some more. A scan of the intro yields some useless sentences and very incomplete descriptions of existing technology. They only mention biometric security as a straw man and completely neglect the use of options such as two-factor authentication and tokens; both have been in use for decades and are cost effective. Further in, the pattern continues which throws everything that follows into a doubtful light. Color is an interesting idea but really is nothing new, just a variation. Skip. There might be interesting references worth following up on.

Kang and Kang \cite{kang_ID_2016} published in PLOS, a journal that evokes different reactions for different people. It's open source and strives for quick turn around on publishing. Research on this journal a few weeks ago sowed that they have adapted to their complaints and made improvements in their process to address them. Since the first two did not warrant a deeper read, my trust in the value of this exercise suggests there might be treasure buried in this article. The abstract is weak and does not really tell me what was accomplished, certainly not much more than what the title suggests. The introduction is better writing than the other two but actually doesn't say much that isn't addressed better in related work. It contains typos so maybe I take back my feelings on the quality of their reviewers. This reads like and introductory paper and I'm not sure who the intended audience is. I don't really see anything that appears novel in the paper. It looks like a lab report for an upper division CS student. I wouldn't even rate it capstone or thesis material for that student. I think the bibliography would be worth harvesting for useful future reading. I don't have any issues with the science, but with the fact that it was submitted and accepted.
\section{Reading Critically and Creatively}
\subsection{}
\citeauthor{gan_regulation_2019} More research into cell mechanisms on the quest to combat viral activity. Meaningful problem, well stated. They avoided declaring well-understood concepts (disease bad, thwarting them good). Authors are very direct and methodical, even pace. Intro references a great deal of the team's work but goes into others when analyzing their results. Their work shows research in legionalla is extensive in well-known journals, some work on pertussis. Good use of figures, I might not have put so much mass spec and gels, I trust their characterization of their mutants. Nice illustrations of calmodulin binding sites. The activity of proteins in cells is critical in understanding how to keep cells healthy. Some tweaks are more obvious than others and some yield more information than others.

There are a number of parallels in CS research. One is direct in studying primary, secondary, tertiary and quaternary structure by simulation or as information science. The methods often parallel in biochemistry with those in neural nets and machine learning. You start with the wild-type protein and make educated mutations and see the increase or decrease in efficacy or information (are you closer or farther?) Rinse, repeat. One yet another level the types of solutions or direction of research can give each other ideas. Folding a protein is like processing data to a correct solution. Both can be done by checking every variation but that doesn't work. Could mechanisms that assist protein folding (like chaperones) be created to reduce those variations to a calculable amount. Much as using a dictionary can greatly reduce the time needed to crack a password file. (see personal notes on protein folding, brain crenelation, neural networks. Can we fold information? How do we chaperone this folding?)

I would like to write a submission in a style like this. Very thorough, very readable. Def keep for structural reference. It shakes loose a number of research paths for others because it touches quite a number of research areas like immunology, physical biochemistry etc.
\subsection{}
\citeauthor{tong_committed_2019} Despite various agreements the planned increases in global infrastructure will continue to increase our overall carbon footprint. USA and EU and China esp electrical are big sources.

Methodically works through the numbers to get to a no nonsense conclusion. Puts it in context without being alarmist (very clear who the audience is.) Clear on what its addressing-major offenders, and what it is leaving unaddressed. No real surprises in the paper. Great figures that encapsulate large amounts of data visually; especially committed emissions per value. Great references as with any Nature pub. Sponsored by NASA, legit. It would be nice to do that someday. Solid math, some I might use again to solve complex math involving growth and reduction. Solid estimates on growth based on history and verified projects. Hard to reduce emissions when increasing production

Critical infrastructure is a current interest. The stability is a factor in homeland security. For example blockchain has a massive carbon footprint. What price will be paid for security through this method? Is it work it? Does the US lose credibility if we don't do our part to reduce greenhouse emissions? If \(CO_2\) levels continue to rise, can we afford the wetter winters, the hotter, dryer summers, the increased power and frequency of hurricanes as the planet tries to correct itself? I could go on but I have in other papers on CI and will in future papers on security.
\nocite{barkal_cd24_2019} \nocite{beaumont_peer_2019} \nocite{buckley_macrophages_2019} \nocite{bhogaraju_inhibition_2019} \nocite{brown_dont_2019}  \nocite{castelvecchi_chemists_2019} \nocite{castelvecchi_dark-energy_2019} \nocite{de_goffau_human_2019} \nocite{dimas_new_2019} \nocite{draaisma_lithium:_2019} \nocite{feldman_barrett_survival:_2019} \nocite{fennell_cancer-cell_2019} \nocite{figgatt_parallel_2019} \nocite{fischer_i_2019} \nocite{gao_dietary_2019} \nocite{gibney_astronomers_2019} \nocite{gong_modulation_2019} \nocite{guglielmi_mexican_2019} \nocite{guillot_signs_2019} \nocite{hessels_x_2019} \nocite{holder_metamorphism_2019} \nocite{holstein_what_2019} \nocite{jiang_plant_2019} \nocite{jonas_pandemic_2019} \nocite{lambert_scientists_2019} \nocite{koepsell_imaging_2019} \nocite{kiser_living_2019} \nocite{lee_activation_2019} \nocite{lewis_head_2019} \nocite{lambert_scientists_2019} \nocite{liu_dynamic_2019} \nocite{liu_formation_2019} \nocite{locke_exome_2019} \nocite{lu_global_2019} \nocite{mahmoudi_tie_2019} \nocite{masefield_use_2019} \nocite{masterson_mosquitoes_2019} \nocite{neilson_astronomy_2019} \nocite{pinto_da_costa_danger_2019} \nocite{ravi_fast_2019} \nocite{reardon_c-section_2019} \nocite{ronsmans_sand_2019} \nocite{schiermeier_eat_2019} \nocite{segata_no_2019} \nocite{steinhorst_how_2019} \nocite{stoevenbelt_rule_2019} \nocite{topol_reproduction_2019} \nocite{witze_astronomy_2019} \nocite{wu_intercellular_2019}

\bibliography{Journal2}
\bibliographystyle{IEEEtranN}
%\bibliographystyle{plain}
\end{document}
