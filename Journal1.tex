\documentclass[journal]{IEEEtran}
\usepackage[numbers]{natbib}
\usepackage[english]{babel}
\usepackage[utf8]{inputenc}
\usepackage{amsmath}
\usepackage{textcomp}
\usepackage{graphicx}
\usepackage{hyperref}
\usepackage{wrapfig}
\usepackage{csquotes}
\usepackage{type1cm}
\usepackage{lettrine}

\newcommand{\myname}{Jennifer S. Smith}
\title{The Life and Research of \myname}
\author{\myname\\
Department of Computer Science\\School of Engineering and Applied Science\\University of Colorado at Colorado Springs\\Colorado Springs, Colorado 80918\\Email: \href{mailto:jsmith@uccs.edu}{jsmith60@uccs.edu}}

\begin{document}
\maketitle
\begin{abstract}
I am pursuing the Engineering-Security PhD and the graduate certificate in Homeland Security and Emergency Management. During this course I would like to explore some of the notions I've had over the past two years of study and see where they go. Previous classes grade on a rubric that does very little to provide feedback about merit and quality of thought. There are two areas I would like to take this time to explore. The first is cognitive bias and artificial intelligence and the second is the intersection on cybersecurity in the space domain. The former would do well as a survey paper and I'd like to create a poster for the Women in Cybersecurity conference with the latter. 
\end{abstract}
\section{Introduction}
\lettrine{M}{s} Smith is a security professional whose experience follows the evolution of multiple convergent fields. She remains focused on community involvement and employment that serves the greater welfare of those around her. She has recently completed a Master's degree in Cybersecurity and Leadership at the University of Washington. Her capstone project assessed methods and viability of cross-walking risk management frameworks.
\subsection{Professional History}
\begin{wrapfigure}{r}{0.21\textwidth}%
\centering
\includegraphics[width=1.5in,clip,keepaspectratio]{HeadShot}
\caption{\myname}
\end{wrapfigure}
After her undergraduate, she worked for JavaSoft in corporate marketing, Lycos in engineering, C-Innovation in research and development, and most recently in network operations for a Ochsner Health System in south eastern Louisiana. She took a hiatus in the 2000s to serve her country and worked primarily for the Naval Space and Warfare Command. During that time she was a trainer onboard multiple ships while underway, trainer and repair technician of EOD robots both stateside and in Iraq \cite{Everett}, \cite{Weiner}, worked in Fleet Support, Navy Reserve Forces Command, and served on various tactical teams to provide services for the Department of Defense and other governmental organizations. She is a certified repair technician for the primary man-transportable robotic systems used by those entities. Additionally she spent several years in a leadership role as a communications subject matter expert in the SeaBees.
\subsection{Additional History}
Ms Smith was born and raised in the San Francisco East Bay and holds degrees in chemistry and cybersecurity. She is published in both the arts and the sciences: Journal of Molecular Biology \cite{JSmith3}, Biochemistry \cite{JSmith1}, \cite{JSmith2}, Redwood Review, and Doorknobs and Bodypaint. She also enjoys teaching and public speaking and has given talks on computing, international affairs, physical biochemistry, and robotics. Her personal awards include a Navy and Marine Corps Commendation Medal, three Navy and Marine Corps Achievement Medals, three Navy Reserve Meritorious Service Medals, and the Armed Forces Reserve Medal with M and bronze hourglass.

Ms. Smith currently resides in DuPont, WA and is excited about finally pursuing her Olympic goals through the sport of curling. She enjoys kayaking, hiking, and urban exploration.
\section{Document Source and Process}
This journal was edited in Overleaf beginning with a template. Code fragments were pulled from the IEEE template and a few others provided by the host to take advantage of prewritten formatting. The .tex file was saved on \href{https://github.com/morisey/UCCS/blob/master/Journal1.tex}{GitHub}. The search results from the Kraemer Library were easy to save in Zotero but the references found at the Center for Homeland Security and Defense were copied, pasted, and edited from their full abstract links. Many of those were missing critical items like the title and source. Zotero exports also correctly add curly braces to maintain capitalization between journal standards. It also produces a label for each of your entry which will come in handy over the years if I read more by the same authors and the number of papers exceed that of easy recall.

The most difficult portion of the assignment was rendering the bibliography. Fortunately, Overleaf jumps over the usual hurdle of users not compiling in the correct order by managing the process automatically. After struggling with the examples found for bibtex and biblatex, natlib resulted in instant results the were personally far easier to debug. Conflicts between packages used in bibtex and biblatex were likely a source of confusion early in the process.

TexStudio is a tool that might prove useful in the future in case there are any issues with the cloud options e.g. if GitHub or Overleaf experiences outage or security breech. It has been installed locally with the MikTex libraries for a rainy day and experimentation in a more manual approach.
\subsection{Research-Related Reading}
\begin{enumerate}
    \item After listening to the audio book of Michael Lewis\textquotesingle s \enquote*{The Undoing Project} it became apparent that many of the widely recognized types of cognitive bias were explored and defined by a small number of researchers. Nobel Laureate, Daniel Kahneman wrote a book called \enquote*{Thinking Fast and Slow} \cite{KahnemanTFS} describing the history of his research and collaborative environment through the years. A key distinction in human thought is between System 1 and System 2 thinking. This prompted a personal interest in how these concepts play with Artificial Intelligence (AI). If the AI is trying to duplicate human though, is it necessary to introduce aspects of System 1 thinking? And, at the other end, can AI reduce human error by promoting System 2 thinking? How is cognitive bias treated by various developers?
    \item Rigas \textit{et al.} \cite{Rigas} is used as an example of student AI survey paper accepted by IEEE. Their paper uses a tree diagram to illustrate the electric vehicle (EV) research landscape which enables the authors to easily point out the overlaps between leaves of the trees. The routing of EVs and the availability of charging stations is critical to their success and the reduction of carbon monoxide, a factor in climate change. From their assessment they recommend that rerouting and incentivizing drivers toward less congested charging stations will increase the chances of success.
    \item The Government Accountability Office of the U.S. released a testimonial from their chief scientist of applied research and methods (Persons) about AI. The testimony identifies four \enquote*{high-consequence} sectors; cybersecurity, automated vehicles, criminal justice and financial services. AI is becoming a key factor in health care and it is strange this was neglected. Persons quotes a technology assessment forum participant as stating, \enquote*{AI can help prevent inappropriate or harmful human bias.} \cite{GAO} This paper offers one way to develop a taxonomy of AI and cognitive bias interplay. It also clearly states the value of AI and the challenges it faces in terms of barriers, access, statutory requirements, and ethical frameworks. Just as cybersecurity is now recognized as a critical step in software development, the latter factors will also need to be incorporated during development, not after deployment.
    \item Internationalization and localization of AI asks some interesting questions about morality when making driving decisions. Rhim \textit{et al.} have defined three human moral reasoning types, moral altruist, moral non determinist, and moral deontologist. This paper is useful for policy makers wishing to understand how different populations view their driving decisions. Meaning one population would sacrifice themselves in an unavoidable accident and another might choose a pedestrian over a school bus. This provides further granularity past the System 1 and 2 dichotomy.
    \item The impact of source factors on situational assessment is explored with the Reliability Game \cite{de_rosa_analytical_2019}. By using war games the researchers are able to expose latent constructs such as source reliability and how they can be converted to subjective probabilities. The equation below 
    \begin{equation}
    p(X_1,\ldots,X_n) = \prod_{i=1}^{N} p(X_i|pa(X_i))
    \end{equation}
    represents the probability distribution representation of the Baysian Network nodes. This is not used to model the real cognitive process but to create computational model and while it appears fancy is it simply a mathematical representation of compounded probabilities of mapped BN parent and child nodes. Each decision in chain adds on to the final probability of a given course of action.
\end{enumerate}
\nocite{AIPanel} \nocite{WeedenCan} \nocite{GAO} \nocite{WeedenSpace} \nocite{Rigas} \nocite{Fidler} \nocite{Stephens} \nocite{Frierson} \nocite{Astorino} \nocite{Colby} \nocite{OSTNSS} \nocite{Saran} \nocite{Cleveland} \nocite{acrecampaign} \nocite{klarin_decade-long_2020} \nocite{porayska-pomsta_blending_2018} \nocite{gheondea-eladi_patient_2019} \nocite{zhang_detecting_2019} \nocite{she_qos-aware_2019}  \nocite{rhim_human_2020} \nocite{linkevicius_linking_2019} \nocite{tanaka_disease_2018} \nocite{shahid_applications_2019}

\bibliographystyle{IEEEtranN}
\bibliography{Journal1}
\end{document}
